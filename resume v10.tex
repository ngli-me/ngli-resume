\documentclass[10pt]{article}

\input{glyphtounicode}

\pdfgentounicode=1

\usepackage{tgpagella}

%-----------------------------------------------------------
%Margin setup

\setlength{\voffset}{0.1in}
\setlength{\paperwidth}{8.5in}
\setlength{\paperheight}{11in}
\setlength{\headheight}{0in}
\setlength{\headsep}{0in}
\setlength{\textheight}{11in}
\setlength{\textheight}{9.5in}
\setlength{\topmargin}{-0.25in}
\setlength{\textwidth}{7in}
\setlength{\topskip}{0in}
\setlength{\oddsidemargin}{-0.25in}
\setlength{\evensidemargin}{-0.25in}

%-----------------------------------------------------------
% I don't like how hyphenating looks
\usepackage[none]{hyphenat}
%-----------------------------------------------------------
\usepackage{enumitem}
\usepackage{titlesec}
\usepackage{multirow}
\pagestyle{empty}
\raggedbottom
\raggedright	
\setlength{\tabcolsep}{0in}

%-----------------------------------------------------------
%Custom commands
\newcommand{\subheadingf}[4]{
\begin{tabular*}{7in}{l@{\extracolsep{\fill}}r}
	\textbf{#1} & #2 \\
	#3 & #4 \\
\end{tabular*}\vspace{1pt}}

\newcommand{\headingtitle}[3]{
\begin{tabular*}{7in}{l@{\extracolsep{\fill}}r}
	\multirow{2}{*}{#1}&#2\\
	&#3\\
\end{tabular*}}

\titleformat{\section}
	{\Large\bfseries\raggedright}
	{}{0em}
	{}
	[\titlerule]

\newcommand{\tabu}[2]{
	\begin{tabular}[t]{ l l }
		#1 & #2
	\end{tabular}}

%-----------------------------------------------------------

\begin{document}
	
	\headingtitle{\textbf{\huge Michael Liang Li}}{mlli1@asu.edu · (480) 363-7999 · 4522 S Boulder St. Gilbert, AZ}{github.com/saesus · instagram.com/saesus}
	\vspace{-2em}
	
	\section{Experience \& Technical Skills}
		\tabu
		{\begin{minipage}[t]{0.3\linewidth}
				\textbf{ImageSTEM} \\
				ASU School of Arts Media \& Engr, Student Worker\\
				September 2020 – Present\\
		\end{minipage}}
		{\begin{minipage}[t]{.7\linewidth}
				\begin{itemize}[noitemsep, topsep=0pt]
					\item Designed modules to introduce artificial intelligence into grades 6-8 curriculum via computational cameras and the medium of Google Colab
					\item Implemented several computation photography techniques including direct and global lighting separation and image segmentation
					\item Researched relighting techniques using neural networks
				\end{itemize}
		\end{minipage}}	
	
		\tabu
		{\begin{minipage}[t]{0.3\linewidth}
				\textbf{Software Intern} \\
				Acme Aerospace, Tempe AZ\\
				August 2019 - April 2020\\
		\end{minipage}}
		{\begin{minipage}[t]{.7\linewidth}
				\begin{itemize}[noitemsep, topsep=0pt]
					\item Scripted in Python and Windows batch to automate checking proper documentation, archiving and committing past projects to VisualSVN from Smartbear Collaborator
					\item Created scripts to log VisualSVN commits into the corresponding JIRA issues
				\end{itemize}
		\end{minipage}}
	
		\tabu
		{\begin{minipage}[t]{0.3\linewidth}
				\textbf{Convolutional Neural Networks} \\
				ASU, CSE494\\
				August 2019 - December 2019\\
		\end{minipage}}
		{\begin{minipage}[t]{.7\linewidth}
				\begin{itemize}[noitemsep, topsep=0pt]
					\item Built the LeNet-5 architecture in Keras, achieving 90\% test accuracy using the MNIST and Fashion MNIST datasets
					\item Reached 75\% test accuracy on the CIFAR-10 Dataset with 10 epochs
				\end{itemize}
		\end{minipage}}
		\tabu
		{\begin{minipage}[t]{0.3\linewidth}
				\textbf{CTF Challenges} \\
				ASU, CSE466\\
				August 2019 - December 2019\\
		\end{minipage}}
		{\begin{minipage}[t]{.7\linewidth}
				\begin{itemize}[noitemsep, topsep=0pt]
					\item Executed a variety of exploit techniques in a lab setting
					\item Reverse engineered x86\_64 (AMD64) Linux binaries with Ghidra
					\item Wrote assembly code to exploit Linux systems and exploiting common insecure programming practices
					\item Testing Shell and Python scripts to attack vulnerabilities with brute force
				\end{itemize}
		\end{minipage}}	
		\tabu
		{\begin{minipage}[t]{0.3\linewidth}
				\textbf{Abstract Syntax Tree} \\
				ASU, CSE340\\
				January 2019 – May 2019\\
		\end{minipage}}
		{\begin{minipage}[t]{.7\linewidth}
				\begin{itemize}[noitemsep, topsep=0pt]
					\item Developed an Abstract Syntax Tree to recognize a grammar, and parse for syntax and semantic errors
					\item Designed with scalability and easy modification in mind, using object-oriented design principles such as abstract base classes
				\end{itemize}
		\end{minipage}}	
		\tabu
		{\begin{minipage}[t]{0.3\linewidth}
				\textbf{Huffman Code} \\
				ASU, CSE310\\
				August 2018 – December 2018\\
		\end{minipage}}
		{\begin{minipage}[t]{.7\linewidth}
				\begin{itemize}[noitemsep, topsep=0pt]
					\item Implemented a Huffman compression algorithm in C++ to generate a tree as well as calculating the compression ratio.
					\item Tested and benchmarked average compression ratio and runtime.
				\end{itemize}
		\end{minipage}}	
	
	\begin{tabular*}{7in}{l@{\extracolsep{\fill}}r}
		\textbf{Proficient in Python, C, C++, Java, Ubuntu Linux} & \textbf{Familiar with x86\_64 assembly, cron, SQL}
	\end{tabular*}
	
	\section{Education}
	\subheadingf{Arizona State University}{Tempe, AZ}
	{Bachelor of Science, Computer Science (GPA: 3.63)}{Expected: May 2021}
	\\Minor in Studio Art\\
	Barrett, the Honors College\\

	\section{Volunteering and Community Involvement}
		\tabu
		{\begin{minipage}[t]{0.3\linewidth}
				\textbf{Secretary/Race Team Member} \\
				ASU Dragon Boat\\
				August 2017 – September 2019\\
		\end{minipage}}
		{\begin{minipage}[t]{.7\linewidth}
				\begin{itemize}[noitemsep, topsep=0pt]
					\item Competed in San Diego, Tempe and Long Beach paddling competitions (2017-2018)
					\item Volunteered with the Iron Man organization as part of the security team, and with Boeing to clean and renovate the Desert Sun Child Development Center
				\end{itemize}
		\end{minipage}}	
		
\end{document}
